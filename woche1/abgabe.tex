\documentclass{homework}
\usepackage{tikz}

\name{Tung Nguyen(ga7led)}
\term{WiSe20-21}
\course{Algebra}
\hwnum{Woche 1}

\begin{document}

\begin{problem}
    Die Kleinsche Vierergruppe
\end{problem}

\begin{solution}
    \begin{enumerate}[label=\alph*)]
        \item \[
            \begin{tabular}{c |c c c c}
            ~   & e(1,1)   & a(-1,1)  & b(1,-1)  & c(-1,-1)  \\
            \hline\vrule height 12pt width 0pt
            e(1,1)   & (1,1)   & (-1,1)   & (1,-1)   & (-1,-1)  \\
            a(-1,1)   & (-1,1)   & (1,1)   & (-1,-1)   & (1,-1)  \\
            b(1,-1)   & (1,-1)   & (-1,-1)   & (1,1)   & (-1,1)  \\
            c(-1,-1)   & (-1,-1)   & (1,-1)   & (-1,1)   & (1,1)  \\
            \end{tabular}
        \]
        \item Die Untergruppen sind :
        \begin{itemize}
            \item A=$\{sdfsd\}$
        \end{itemize}
    \end{enumerate}
\end{solution}

\end{document}
